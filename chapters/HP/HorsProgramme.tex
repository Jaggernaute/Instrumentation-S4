\chapter{Hors programme}
\label{chap:hors-programme}

\section{Echantillonage}

L'operation d'\'echantillonage consiste a representer un signal fonction du temps 
\(s(t)\) \`a des instants multiples d'un intervalle de temps \(T\) (p\'eriode d'\'echantillonage).
Cette opoeration s'analyse simplement par l'intermediaire de la th\'eorie des distributions.
Par definition la distribution des masses unitaires aux points de l'axe r\'eel multiples 
entiers de la periode \(T\), assosiee a la fonction \(s(t)\) a l'ensemble de ses valeurs
\(s(nT)\) o\`u \(n\) est un entier.
\[
    u(t) = \sum_{n=-\infty}^{+\infty} \delta(t - nT)
\]

Soit $\mathbb{R}^n$. La ``fonction delta de Dirac'', notée $\delta$, n’est pas 
une fonction ordinaire, mais une mesure borélienne particulière. Elle est 
\textbf{concentrée en un seul point}, à savoir l’origine $0$, et attribue une 
masse totale égale à $1$ à ce point. On la définit ainsi comme la mesure de 
probabilité telle que :

\[
\delta(\{0\}) = 1, \quad \text{et plus généralement :} \quad \delta(A) = 
\begin{cases}
1 & \text{si } 0 \in A \\
0 & \text{sinon}
\end{cases}
\quad \text{pour tout borélien } A \subset \mathbb{R}^n.
\]

Par abus de langage, on dit que la fonction delta de Dirac est 
\textbf{nulle partout sauf en 0}, où elle prend une valeur infinie, de manière à 
ce que \textbf{l’intégrale sur l’ensemble de l’espace soit égale à 1}. Il ne 
s’agit pas d’une fonction au sens classique, mais d’un objet distribué, 
c’est-à-dire une \textbf{mesure} (ou une \textbf{distribution}) dont l’effet se 
fait uniquement sentir en un seul point.

La mesure $\delta$ est une \textbf{mesure de Radon}, ce qui signifie en 
particulier qu’elle est définie sur les ensembles boréliens et qu’elle est finie 
sur les compacts. Elle peut alors être identifiée à une 
\textbf{forme linéaire continue de norme 1} sur l’espace des fonctions continues 
à support compact, noté $\mathcal{C}_c(\mathbb{R}^n)$. Cette forme linéaire agit 
comme suit :

\[
\mathcal{C}_c(\mathbb{R}^n) \to \mathbb{R}, \quad \varphi \mapsto \langle \delta, \varphi \rangle := \int \varphi \, \mathrm{d}\delta = \varphi(0)
\]

Autrement dit, \textbf{intégrer une fonction test $\varphi$ contre la mesure 
$\delta$ revient simplement à évaluer cette fonction en $0$}.

L’espace $\mathcal{C}_c(\mathbb{R}^n)$ désigne l’ensemble des fonctions 
continues à support compact dans $\mathbb{R}^n$, muni de la norme uniforme 
$\|\cdot\|_\infty$ associée à la convergence uniforme.

