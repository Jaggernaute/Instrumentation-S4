\chapter{Hors programme}\label{chap:hors-programme}

\section{Échantillonnage}\label{hp:echantillonage}

L’opération d’échantillonnage consiste à représenter un signal fonction du temps 
\(s(t)\) \`a des instants multiples d’un intervalle de temps \(T\) (p\'eriode d’échantillonnage).
Cette op\'eration s'analyse simplement par l’intermédiaire de la th\'eorie des distributions.
Par définition la distribution des masses unitaires aux points de l'axe r\'eel multiples 
entiers de la période \(T\), associée à la fonction \(s(t)\) a l'ensemble de ses valeurs
\(s(nT)\) o\`u \(n\) est un entier.
\[
    u(t) = \sum_{n=-\infty}^{+\infty} \delta(t - nT)
\]
% \textit{Pour plus de details, voir \Cref{sec:dirac}.}

\section{Fonction delta de Dirac}
\label{sec:dirac}

Soit $\mathbb{R}^n$. La fonction delta de Dirac, notée $\delta$, n’est pas 
une fonction ordinaire, mais une mesure borélienne\footnote{Glossaire : \gls{mesure_borelienne}} particulière. Elle est 
\textbf{concentrée en un seul point}, à savoir l’origine $0$, et attribue une 
masse totale égale à $1$ à ce point. On la définit ainsi comme la mesure de 
probabilité telle que :

\begin{align*}
    \delta(\{0\})  &= 1\\
    \text{et plus généralement :} \quad \delta(A) &=
    \begin{cases}
        1 & \text{si } 0 \in A \\
        0 & \text{sinon}
    \end{cases}
\end{align*}
Pour tout borélien\footnote{Glossaire : \gls{borelien}} \(A \subset \mathbb{R}^n\).

Par abus de langage, on dit que la fonction delta de Dirac est 
\textbf{nulle partout sauf en 0}, où elle prend une valeur infinie, de manière que 
\textbf{l’intégrale sur l’ensemble de l’espace soit égale à 1}. Il ne 
s’agit pas d’une fonction au sens classique, mais d’un objet distribué, 
c’est-à-dire une \textbf{mesure} (ou une \textbf{distribution}) dont l’effet se 
fait uniquement sentir en un seul point.

La mesure $\delta$ est une \textbf{mesure de Radon}, ce qui signifie en 
particulier qu’elle est définie sur les ensembles boréliens et qu’elle est finie 
sur les compacts. Elle peut alors être identifiée à une 
\textbf{forme linéaire continue de norme 1} sur l’espace des fonctions continues 
à support compact, noté $\mathcal{C}_c(\mathbb{R}^n)$. Cette forme linéaire agit 
comme suit :

\[
\mathcal{C}_c(\mathbb{R}^n) \to \mathbb{R}, \quad \varphi \mapsto \langle \delta, \varphi \rangle := \int \varphi \, \mathrm{d}\delta = \varphi(0)
\]

Autrement dit, \textbf{intégrer une fonction test $\varphi$ contre la mesure 
$\delta$ revient simplement à évaluer cette fonction en $0$}.

L’espace $\mathcal{C}_c(\mathbb{R}^n)$ désigne l’ensemble des fonctions 
continues à support compact dans $\mathbb{R}^n$, muni de la norme uniforme 
$\|\cdot\|_\infty$ associée à la convergence uniforme.

\begin{figure}[!ht]
    \centering
    \begin{tikzpicture}
        \begin{axis}[
            grid=both,
            domain=0:2, samples=100,
            ytick={-0.2,0,1,1.2},
            xtick={-1.2,-1,0,1,1.2},
            ymin=-0.2, ymax=1.2,
            xmin=-1.2, xmax=1.2,
        ]
            \addplot[thick, RoyalPurple, samples=100, domain=-1.2:1.2] {0};
            \draw[->, thick, RoyalPurple] (axis cs:0,0) -- (axis cs:0,1);
        \end{axis}
    \end{tikzpicture}
    \caption{La fonction delta de Dirac.}
    \label{figDirac}%
\end{figure}

\emph{\Cref{figDirac} La flèche signifie que toute la "masse" de la fonction est concentrée en 0 et vaut 1.}


\subsection{Le peigne de Dirac}
\label{sec:peigne-dirac}
Le peigne de Dirac est une mesure borélienne sur $\mathbb{R}^n$ qui
\textbf{concentre une masse unitaire} sur chaque point de la grille
\(\mathbb{Z}^n\) (l’ensemble des entiers naturels). Il est défini par la
\textbf{somme infinie} de la fonction delta de Dirac, notée $\delta$,
\textbf{pondérée} par la période d’échantillonnage $T$ :
\[
    \Sha_T (t)\ \stackrel{\mathrm{def}}{=}\  \sum_{k=-\infty}^{\infty}\delta_{k T}(t) = \sum_{k=-\infty}^{\infty}\delta(t-kT).
\]

\begin{figure}[!ht]
    \begin{tikzpicture}
        \begin{axis}[
            grid=both,
            domain=-1:1, samples=100,
            ytick={-0.2,0,1,1.2},
            xtick={-1.2,-1,0,1,1.2},
            xticklabels={,\(-T\),0,\(T\),},
            yticklabels={,0,1,},
            ymin=-0.2, ymax=1.2,
            xmin=-1.2, xmax=1.2,
        ]
            \addplot[thick, RoyalPurple, samples=100, domain=-1.2:1.2] {0};
            \pgfplotsinvokeforeach {-1,0,1}{%
                \draw[->, thick, RoyalPurple] (axis cs:#1,0) -- (axis cs:#1,1);
            }
        \end{axis}
    \end{tikzpicture}
    \caption{Le peigne de Dirac.}
    \label{fig:peigne-dirac}%
\end{figure}

\subsection{Série de Fourier}

Il est clair que \(\Sha_T(t)\) est périodique de période \(T\). C’est-à-dire que :
\[
    \Sha_T(t+T) = \Sha_T(t)
\]
pour tout \(t\). La s\'erie de Fourrier complexe de ce genre de fonction p\'eriodique est :
\[
    \Sha_{T}(t)=\sum_{n=-\infty}^{+\infty}c_{n}e^{i2\pi n{\frac{t}{T}}}
\]
o\`u les coefficients de Fourier sont donn\'es par :
\begin{align*}
c_n &= \frac{1}{T} \int_{t_0}^{t_0+T}\Sha_{T}(t)e^{-i2\pi n\frac{t}{T}}\,dt\quad (-\infty < t_0 < +\infty )\\
    &= \frac{1}{T} \int_{-\frac{T}{2}}^{\frac{T}{2}} \Sha_{T}(t) e^{-i2\pi n\frac{t}{T}}\,dt\\
    &= \frac{1}{T} \int_{-\frac{T}{2}}^{\frac{T}{2}} \delta(t) e^{-i2\pi n\frac{t}{T}}\,dt\\
    &= \frac{1}{T} e^{-i2\pi n\frac{0}{T}}\\
    &= \frac{1}{T}
\end{align*}

Tous les coefficients de Fourier sont donc égaux à \(\frac{1}{T}\). En
conséquence, la série de Fourier de \(\Sha_{T}(t)\) est :
\[
    \Sha_{T}(t)=\sum_{n=-\infty}^{+\infty}\frac{1}{T}e^{i2\pi n{\frac{t}{T}}}
\]
Quand la p\'eriode est \(T=1\), on obtient :
\[
    \Sha(x)=\sum_{n=-\infty}^{+\infty}e^{i2\pi n x}
\]

Cette s\'erie diverge, quand elle est \'etudi\'ee comme une s\'erie de nombres complexes 
ordinaires. Mais elle converge dans le sens des distributions.

Une "racine carr\'ee" du peigne de Dirac a des applications en physique, sp\'ecifiquement :

\[
    \delta_N^{(1 / 2)}(\xi) = \frac{1}{\sqrt{NT}} \sum_{\nu=0}^{N-1} e^{-i \frac{2\pi}{T}\xi \nu}, \quad 
\lim_{N \rightarrow \infty}\left|\delta_N^{(1 / 2)}(\xi)\right|^2= \sum_{k=-\infty}^{\infty} \delta(\xi - kT)
\]
m\^eme si ce n'est pas une distribution au sens de classique.

\section{Transformée de Fourier}

La transformée de Fourier d’un peigne de Dirac est également un peigne de Dirac. 
Pour la transformée de Fourier \(\mathcal{F}\), exprimée dans le domaine des 
fréquences (en \si{\hertz} ), le peigne de Dirac \(\Sha_T\) de période \(T\) se 
transforme en un peigne de Dirac redimensionné de période \(1/T\), c’est-à-dire que :
\begin{align*}
    \mathcal{F}\left[f\right]\left(\xi\right) &= \int_{-\infty}^{+\infty} f(t) e^{-i 2 \pi \xi t} dt\\
    \mathcal{F}\left[\Sha_T\right]\left(\xi\right) &= \frac{1}{T}\sum_{k=-\infty}^{+\infty} \delta\left(\xi - k\frac{1}{T}\right)\\
    &= \frac{1}{T}\Sha_\frac{1}{T}\left(\xi\right)
\end{align*}

Est proportionnelle à un autre peigne de Dirac de période \(1/T\) (radians par seconde).
Le peigne de Dirac \(\Sha\) de période \(T=1\) est une fonction propre de \(\mathcal{F}\) 
vers la valeur propre \(1\).

Ce résultat peut être établi en considérant la transformée de Fourier 
\(S_\tau\left(\xi\right)=\mathcal{F}\left[s_\tau\right]\left(\xi\right)\) de la famille
de fonctions \(s_\tau(x)\) de la forme :

