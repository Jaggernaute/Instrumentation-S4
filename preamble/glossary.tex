% RFID
\newglossaryentry{rfid}{
    name={RFID},
    description={
        La technologie \gls{rfid_acro} utilise des ondes radio pour identifier des objets à distance. 
        Elle est utilisée dans de nombreux domaines tels que la logistique, la distribution, la santé, la sécurité, etc.
    }
}

% IOT
\newglossaryentry{iot}{
    name={IoT},
    description={
        L'\gls{iot_acro} est un réseau d'objets physiques, d'appareils et de systèmes connectés à Internet et équipés de capteurs, 
        de logiciels et d'une connectivité réseau, leur permettant de collecter et d'échanger des données.
    }
}

% CPU
\newglossaryentry{cpu}{
    name={CPU},
    description={
        Le \gls{cpu_acro} est le cerveau de l'ordinateur. Il exécute des instructions, 
        traite les données et contrôle les opérations du système.
    }
}

% Mesure Borelienne
\newglossaryentry{mesureBorelienne}{
    name={Mesure borélienne},
    description={
        Une mesure borélienne est une mesure définie sur une \gls{sigma-algebre}  
        de Borel, qui est l'ensemble des ensembles mesurables dans un espace topologique.
    }
}

% sigma-algèbre
\newglossaryentry{sigma-algebre}{
    name={\(\sigma\)-algèbre},
    description={
        Une \textit{\(\sigma\)-algèbre} est une collection d'ensembles qui est fermée sous 
        les opérations de complémentation et d'union dénombrable.
    }
}

% Borelien
\newglossaryentry{borelien}{
    name={Borélien},
    description={
        Un ensemble est dit \textit{borélien} s'il appartient à la \gls{sigma-algebre} de Borel, 
        c'est-à-dire s'il peut être obtenu à partir d'ensembles ouverts par des opérations 
        de complémentation et d'union dénombrable.
    }
}

% mesure
\newglossaryentry{mesure}{
    name={Mesure},
    description={
        Une mesure est une fonction qui attribue une valeur non négative à des ensembles, 
        permettant de quantifier leur taille ou leur volume.
    }
}

\newacronym{rfid_acro}{RFID}{Radio Frequency IDentification}
\newacronym{iot_acro}{IoT}{Internet of Things}
\newacronym{cpu_acro}{CPU}{Central Processing Unit}
